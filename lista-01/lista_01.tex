\documentclass{exam}

%....................................
% PACOTES USADOS
%....................................

\usepackage[portuguese]{babel}
\usepackage[utf8]{inputenc}
\usepackage{geometry}
\usepackage[list-final-separator={ e }, output-decimal-marker={,}]{siunitx}
\usepackage{graphicx}
\usepackage{float}
\usepackage{verbatim}
\usepackage{lmodern}
\usepackage{xpatch}
\usepackage{enumerate}
\usepackage{enumitem}
\usepackage{multicol, setspace}
\usepackage{indentfirst}
\usepackage{parskip}
\usepackage{amssymb, amsmath, amsfonts, dsfont}
\usepackage{makeidx}
\usepackage[sharp]{easylist}

%\usepackage{textcomp}

%....................................
% OUTRAS CONFIGURAÇÕES
%....................................


\pagestyle{headandfoot}

%\runningfootrule

\firstpageheader{\texttt{spholima@gmail.com}}{}{\includegraphics[scale=.9]{by-nc-sa}}
\runningheader{\textsc{Números inteiros}}{}{\includegraphics[scale=.9]{by-nc-sa}}
\firstpagefooter{}
	{\textsc{\thepage}}
	{}
\runningfooter{}
	{\textsc{\thepage}}
	{}

\geometry{a4paper,
	top=2.5cm,
	bottom=2.5cm,
	left=1.5cm,
	right=1.5cm
}
\columnsep=0.8cm

%....................................
% CONFIGURAÇÕES DA CLASSE EXAM
%....................................
\renewcommand{\thequestion}{\bf \arabic{question}}
\renewcommand{\choicelabel}{{\bf (\thechoice)}}
\pointpoints{ponto}{pontos}
\pointformat{[\bf \thepoints]}



%....................................
% INÍCIO DO DOCUMENTO
%....................................

\begin{document}

	\begin{center}
\large{\textsc{Tópico 01 - Números inteiros\\
Março de 2020\\ (em plena crise do corona vírus$\ldots$)}}
	\end{center}

%\maketitle
	
	\begin{multicols*}{2}
	\setlength{\columnseprule}{1pt}


\tableofcontents

\section{Números naturais}

Representado por $\mathbb{N}$, é o conjunto dos números inteiros positivos maiores que zero, incluindo o número zero.

\begin{center}
	$\mathbb{N} = \{0,\, 1,\, 2,\, 3,\, 4,\, 5,\, 6,\, 7,\, 8,\, 10,\, 11,\, 12,\, \ldots\}$
\end{center}

Denota-se por $\mathbb{N^*}$ o conjunto composto apenas pelos números inteiros positivos. Este conjunto, semelhante ao conjunto dos números naturais, difere do primeiro por não conter o número zero em sua lista de elementos.

\begin{center}
	$\mathbb{N^*} = \{1,\, 2,\, 3,\, 4,\, 5,\, 6,\, 7,\, 8,\, 10,\, 11,\, 12,\, \ldots\}$	
\end{center}



\section{Números inteiros}

Representado por $\mathbb{Z}$, é uma extensão dos números naturais, incluindo também os números negativos em sua lista de elementos. Nesse conjunto, para cada número natural positivo, existirá o seu simétrico negativo. Para o $1$, $-1$; para o $2$, $-2$; para o $3$, $-3$ e assim por diante.

\begin{center}
	$\mathbb{Z} = \{\ldots,\, -4,\, -3,\, -2,\, -1,\, 0,\, 1,\, 2,\, 3,\, 4,\, \ldots\}$	
\end{center}

Do mesmo modo que fizemos para o conjunto $\mathbb{N^*}$, denotamos por $\mathbb{Z^*}$ o conjunto dos números inteiros sem a presença do zero em sua lista de elementos.

\begin{center}
	$\mathbb{Z^*} = \{\ldots,\, -4,\, -3,\, -2,\, -1,\, 1,\, 2,\, 3,\, 4,\, \ldots\}$	
\end{center}


Quando o número é \textit{positivo}, ao escrevê-lo, não precisamos obrigatoriamente indicar o seu sinal. Por exemplo, $+20 = 20$.

Já para os números \textit{negativos}, é sempre obrigatório a indicação do sinal.


	\subsection{Adição de inteiros}

		\subsubsection{Inteiros de mesmo sinal}
		
		Para adicionar dois números de mesmo sinal, é preciso somar os valores e atribuir ao resultado o mesmo sinal dos números somados.
		
		\textbf{Exemplos:}
		
		\begin{easylist}[enumerate]
			## $(-15)+(-20)=-35$
			## $(-10)+(-50)=-60$
			## $(-3)+(-2)=-5$
			## $30+40=70$
			## $9+8=17$
			## $(+15)+(+5) = 15+5 = 20$
			## $(-7)+(-3)= -7+(-3)=-10$
						
		\end{easylist}
		

		\subsubsection{Inteiros de sinais diferentes}
		
		Para somar dois números inteiros, de sinais contrários, basta subtrair o maior deles pelo menor e atribuir ao resultado o sinal do maior número.
		
		\textbf{Exemplos:}
		
		\begin{easylist}[enumerate]
			## $(-15)+(+20)=+5$
			## $(-30)+(+20)=-10$
			## $(-50)+(+20)=-30$
			## $(+40)+(-30)=+10$
			## $(-30)+(+40)=+10$			
				
		\end{easylist}


\begin{questions}

	\qformat{\textbf{Questão} \thequestion \dotfill}

\section{Questões para treinamento}


	\question Dê o resultado das operações a seguir entre \textit{inteiros de mesmo sinais}.

	\begin{choices}

		\choice $(-3)+(-6)=$
		\choice $(-3)+(-5)=$
		\choice $(+5)+(+3)=$
		\choice $(-5)+(-4)=$
		\choice $(+16)+(+12)=$
		\choice $(-21)+(-4)=$
		\choice $(-13)+(-33)=$
		\choice $(-32)+(-14)=$
		\choice $(+7)+(+9)=$
		\choice $(-16)+(-22)=$
		\choice $(-3)+(-24)=$
		\choice $(+7)+(+3)=$
		\choice $(-26)+(-19)=$
		\choice $(-29)+(-33)=$
		\choice $(+3)+(+4)=$
		\choice $(-21)+(-11)=$
		\choice $(+34)+(+42)=$
		\choice $(-55)+(-20)=$
		\choice $(+60)+(+30)=$
		\choice $(-7)+(-18)=$

	\end{choices}


	\question Mais exercícios: soma de \textit{inteiros de mesmo sinal}, com alguns parênteses já eliminados.

\begin{choices}
	
	\choice $(-2)+(-5)=$
	\choice $(-4)+(-2)=$
	\choice $(+5)+(+4)=$
	\choice $(-3)+(-4)=$
	\choice $+6 +(+12)=$
	\choice $(-26)+(-3)=$
	\choice $(-14)+(-31)=$
	\choice $(-35)+(-21)=$
	\choice $+5+8=$
	\choice $(-15)+(-18)=$
	\choice $(-2)+(-26)=$
	\choice $+6+9=$
	\choice $-28+(-17)=$
	\choice $(-19)+(-36)=$
	\choice $(+6)+3=$
	\choice $-12+(-13)=$
	\choice $+32+22=$
	\choice $-45+(-10)=$
	\choice $40+30=$
	\choice $(-9)+(-29)=$
	
\end{choices}


	\question Efetue as operações a seguir entre \textit{inteiros de sinais opostos}.

\begin{choices}
	
	\choice $(-3)+(+6)=$
	\choice $(+2)+(-5)=$
	\choice $(+5)+(-4)=$
	\choice $(-5)+(+9)=$
	\choice $(+17)+(-12)=$
	\choice $(+25)+(-7)=$
	\choice $(-17)+(+36)=$
	\choice $(-21)+(+16)=$
	\choice $(-4)+(+8)=$
	\choice $(+12)+(-27)=$
	\choice $(+5)+(-23)=$
	\choice $(+7)+(-4)=$
	\choice $(-21)+(+13)=$
	\choice $(+28)+(-34)=$
	\choice $(-7)+(+4)=$
	\choice $(+22)+(-19)=$
	\choice $(+43)+(-46)=$
	\choice $(-51)+(+30)=$
	\choice $(+50)+(-32)=$
	\choice $(+5)+(-28)=$
	
\end{choices}


	\question Mais exercícios: soma de \textit{inteiros de sinais opostos}.

\begin{choices}
	
	\choice $(-7)+(+3)=$
	\choice $(+3)+(-9)=$
	\choice $(+4)+(-4)=$
	\choice $(+8)+(-11)=$
	\choice $(+15)+(-19)=$
	\choice $(+15)+(-9)=$
	\choice $(-18)+(+26)=$
	\choice $(-23)+(+15)=$
	\choice $(+13)+(-7)=$
	\choice $(+16)+(-14)=$
	\choice $(+8)+(-21)=$
	\choice $(+4)+(-5)=$
	\choice $(-34)+(+18)=$
	\choice $(+32)+(-42)=$
	\choice $(-22)+(+5)=$
	\choice $(+45)+(-18)=$
	\choice $(+38)+(-40)=$
	\choice $(-63)+(+32)=$
	\choice $(+72)+(-33)=$
	\choice $(-16)+(+24)=$
	
\end{choices}

	\newpage
	\question Soma de \textit{inteiros de sinais opostos} (agora sem alguns parênteses desnecessários).

\begin{choices}
	
	\choice $-4 + 7=$
	\choice $3 + (-6)=$
	\choice $6 + (-5)=$
	\choice $-6 + 4=$
	\choice $16 + (-21)=$
	\choice $36 + (-9)=$
	\choice $-15 + 32=$
	\choice $-23 + 29=$
	\choice $-5 + 9=$
	\choice $13 + (-32)=$
	\choice $6 + (-21)=$
	\choice $8 + (-3)=$
	\choice $-33 + 19=$
	\choice $21 + (-32)=$
	\choice $-8 + 5=$
	\choice $42 + (-13)=$
	\choice $73 + (-31)=$
	\choice $-66 + 20=$
	\choice $70 + (-53)=$
	\choice $9 + (-34)=$
	

\end{choices}

	\question Mais exercícios$\ldots$

\begin{choices}
	
	\choice $-23 + 46=$
	\choice $-12 + (-15)=$
	\choice $51 + (-42)=$
	\choice $-15 + 15=$
	\choice $-27 + (-12)=$
	\choice $35 + (-17)=$
	\choice $-47 + 36=$
	\choice $-21 +(-26)=$
	\choice $-4 + 8=$
	\choice $-12 + (-27)=$
	\choice $15 + (-35)=$
	\choice $-17 + (-11)=$
	\choice $-25 + (-13)=$
	\choice $98 + (-64)=$
	\choice $17 + 41=$
	\choice $52 + (-67)=$
	\choice $13 + (-36)=$
	\choice $-82 + (-10)=$
	\choice $-70 + (-22)=$
	\choice $16 + (-37)=$
	
\end{choices}

	\question Vai na fé!

\begin{choices}
	
	\choice $-22 + 62=$
	\choice $-19 + (-25)=$
	\choice $44 + (-33)=$
	\choice $-18 + 21=$
	\choice $-33 + (-18)=$
	\choice $25 + (-13)=$
	\choice $-32 + 51=$
	\choice $-16 + (-27)=$
	\choice $-9 + 13=$
	\choice $-24 + (-50)=$
	\choice $25 + (-52)=$
	\choice $-31 + (-23)=$
	\choice $-42 + (-20)=$
	\choice $73 + (-54)=$
	\choice $28 + 15=$
	\choice $65 + (-12)=$
	\choice $37 + (-75)=$
	\choice $-76 +(-48)=$
	\choice $-20 + (-17)=$
	\choice $39 + (-28)=$

	
\end{choices}


\end{questions}

	\end{multicols*}

\newpage
\section{Gabarito}

\begin{multicols*}{4}
\setlength{\columnseprule}{1pt}
\begin{enumerate}

\item 	a) $-9$ \\
		b) $-8$ \\
		c) $8$ \\
		d) $-9$ \\
		e) $28$ \\
		f) $-25$ \\
		g) $-46$ \\
		h) $-46$ \\
		i) $16$ \\
		j) $-38$ \\
		k) $-27$ \\
		l) $10$ \\
		m) $-45$ \\
		n) $-62$ \\
		o) $7$ \\
		p) $-32$ \\
		q) $76$ \\
		r) $-75$ \\
		s) $90$ \\
		t) $-25$ \\
		
\item 	a) $-7$ \\
		b) $-6$ \\
		c) $9$ \\
		d) $-7$ \\
		e) $18$ \\
		f) $-29$ \\
		g) $-45$ \\
		h) $-56$ \\
		i) $13$ \\
		j) $-33$ \\
		k) $-28$ \\
		l) $15$ \\
		m) $-45$ \\
		n) $-55$ \\
		o) $9$ \\
		p) $-25$ \\
		q) $54$ \\
		r) $-55$ \\
		s) $70$ \\
		t) $-38$ \\

\item 	a) $3$ \\
		b) $-3$ \\
		c) $1$ \\
		d) $4$ \\
		e) $5$ \\
		f) $18$ \\
		g) $19$ \\
		h) $-5$ \\
		i) $4$ \\
		j) $-15$ \\
		k) $-18$ \\
		l) $3$ \\
		m) $-8$ \\
		n) $-6$ \\
		o) $-3$ \\
		p) $3$ \\
		q) $-3$ \\
		r) $-21$ \\
		s) $18$ \\
		t) $-23$ \\

\item 	a) $-4$ \\
		b) $-6$ \\
		c) $0$ \\
		d) $-3$ \\
		e) $-4$ \\
		f) $6$ \\
		g) $8$ \\
		h) $-8$ \\
		i) $6$ \\
		j) $2$ \\
		k) $-13$ \\
		l) $-1$ \\
		m) $-16$ \\
		n) $-10$ \\
		o) $-17$ \\
		p) $27$ \\
		q) $-12$ \\
		r) $-31$ \\
		s) $39$ \\
		t) $8$ \\


\item 	a) $-3$ \\
		b) $-3$ \\
		c) $1$ \\
		d) $-2$ \\
		e) $-5$ \\
		f) $27$ \\
		g) $17$ \\
		h) $6$ \\
		i) $4$ \\
		j) $-19$ \\
		k) $-15$ \\
		l) $5$ \\
		m) $-14$ \\
		n) $-11$ \\
		o) $-3$ \\
		p) $29$ \\
		q) $42$ \\
		r) $-46$ \\
		s) $17$ \\
		t) $-25$ \\


\item 	a) $23$ \\
		b) $-27$ \\
		c) $9$ \\
		d) $0$ \\
		e) $-39$ \\
		f) $18$ \\
		g) $-11$ \\
		h) $-47$ \\
		i) $4$ \\
		j) $-39$ \\
		k) $-25$ \\
		l) $-28$ \\
		m) $-38$ \\
		n) $34$ \\
		o) $58$ \\
		p) $-15$ \\
		q) $-23$ \\
		r) $-92$ \\
		s) $-92$ \\
		t) $-21$ \\


\item 	a) $40$ \\
		b) $-44$ \\
		c) $11$ \\
		d) $3$ \\
		e) $-51$ \\
		f) $12$ \\
		g) $19$ \\
		h) $-43$ \\
		i) $4$ \\
		j) $-74$ \\
		k) $-27$ \\
		l) $-54$ \\
		m) $-62$ \\
		n) $19$ \\
		o) $43$ \\
		p) $53$ \\
		q) $-38$ \\
		r) $-124$ \\
		s) $-37$ \\
		t) $11$ \\

\end{enumerate}
\end{multicols*}




\end{document}