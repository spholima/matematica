\begin{questions}

	\qformat{\textbf{Questão} \thequestion \dotfill}

\section{Questões para treinamento}

\question Efetuas as operações a seguir:
	\begin{choices}

		\choice $(-3)-(-6)=$
		\choice $(-3)-(-5)=$
		\choice $(+5)-(+3)=$
		\choice $(-5)-(-4)=$
		\choice $(+16)-(+12)=$
		\choice $(-21)-(-4)=$
		\choice $(-13)-(-13)=$
		\choice $(-32)-(-14)=$
		\choice $(+7)-(+9)=$
		\choice $(-16)-(-22)=$
		\choice $(-3)-(-24)=$
		\choice $(+7)-(+3)=$
		\choice $(-26)-(-19)=$
		\choice $(-29)-(-33)=$
		\choice $(+3)-(+4)=$
		\choice $(-21)-(-11)=$
		\choice $(+34)-(+42)=$
		\choice $(-55)-(-20)=$
		\choice $(+60)-(+30)=$
		\choice $(-7)-(-18)=$

	\end{choices}

	\question Mais exercícios.

\begin{choices}
	
	\choice $(-7)-(+3)=$
	\choice $(+3)-(-9)=$
	\choice $(+4)-(-4)=$
	\choice $(+8)-(-11)=$
	\choice $(+15)-(-19)=$
	\choice $(+15)-(-9)=$
	\choice $(-18)-(+26)=$
	\choice $(-23)-(+15)=$
	\choice $(+13)-(-7)=$
	\choice $(+16)-(-14)=$
	\choice $(+8)-(-21)=$
	\choice $(+4)-(-5)=$
	\choice $(-34)-(+18)=$
	\choice $(+32)-(-42)=$
	\choice $(-22)-(+5)=$
	\choice $(+45)-(-18)=$
	\choice $(+38)-(-40)=$
	\choice $(-63)-(+32)=$
	\choice $(+72)-(-33)=$
	\choice $(-16)-(-24)=$
	
\end{choices}

\question Mais exercícios: subtração de inteiros.

\begin{choices}
	
	\choice $(-2)-(-5)=$
	\choice $(-4)-(-2)=$
	\choice $(+5)-(+4)=$
	\choice $(-3)-(-4)=$
	\choice $+6 - (+12)=$
	\choice $(-26)-(-3)=$
	\choice $(-14)-(-31)=$
	\choice $(-35)-(-21)=$
	\choice $+5 - 8=$
	\choice $(-15)-(-18)=$
	\choice $(-2)-(-26)=$
	\choice $+6 - 9=$
	\choice $-28 - (-17)=$
	\choice $(-19) - (-36)=$
	\choice $(+6) - 3=$
	\choice $-12 - (-13)=$
	\choice $+32 - 22=$
	\choice $-45 - (-10)=$
	\choice $40 - 30=$
	\choice $(-9) - (-29)=$
	
\end{choices}


\question Mais exercícios.

\begin{choices}
	
	\choice $-22 - 62=$
	\choice $-19 - (-25)=$
	\choice $44 - (-33)=$
	\choice $-18 - 21=$
	\choice $-33 - (-18)=$
	\choice $25 - (-13)=$
	\choice $-32 - 51=$
	\choice $-16 - (-27)=$
	\choice $-9 - 13=$
	\choice $-24 - (-50)=$
	\choice $25 - (-52)=$
	\choice $-31 - (-23)=$
	\choice $-42 - (-20)=$
	\choice $73 - (-54)=$
	\choice $28 - 15=$
	\choice $65 - (-12)=$
	\choice $37 - (-75)=$
	\choice $-76 - (-48)=$
	\choice $-20 - (-17)=$
	\choice $39 - (-28)=$
	
\end{choices}


\question Mais exercícios de subtração.

\begin{choices}
	
	\choice $-23 - 46=$
	\choice $-12 - (-15)=$
	\choice $51 - (-42)=$
	\choice $-15 - 15=$
	\choice $-27 - (-12)=$
	\choice $35 - (-17)=$
	\choice $-47 - 36=$
	\choice $-21 - (-26)=$
	\choice $-4 - 8=$
	\choice $-12 - (-27)=$
	\choice $15 - (-35)=$
	\choice $-17 - (-11)=$
	\choice $-25 - (-13)=$
	\choice $98 - (-64)=$
	\choice $17 - 41=$
	\choice $52 - (-67)=$
	\choice $13 - (-36)=$
	\choice $-82 - (-10)=$
	\choice $-70 - (-22)=$
	\choice $16 - (-37)=$
	
\end{choices}

\section{Problemas}
	
	Se for, vá na paz.

\question Rubens nasceu no ano 92 a.C. e se casou aos 29 anos de idade. Em que ano ele se casou?

\question Se um termômetro marca \SI{9}{\celsius} depois que a temperatura subiu \SI{17}{\celsius}, qual era a temperatura inicial?

\question Um balão subiu 17 quilômetros e, em seguida, desceu 9 quilômetros. A quantos quilômetros o balão se encontra do ponto que ele saiu?

\question Um helicóptero que voa a 510 metros acima do nível do mar, identifica um submarino que se encontra a uma profundidade de 203 metros. A que distância se encontra o submarino do helicóptero?

\question De um depósito, que contém 800 litros de água, se retiram 240 litros e, logo após, acrescentam 250 litros. Depois, se retiram 180 litros e acrescentam $x$ litros. Qual é o valor de $x$ se ao final o depósito contém 500 litros de água?

\question Hélder e Laura partem de um mesmo lugar de bicicleta. Hélder avança 7 quilômetros e logo retrocede 2 quilômetros. Já Laura, avança 7 quilômetros e retrocede 1. Ao final, a que distância se encontra um do outro?

\question Em uma cidade ao norte da Rússia, a temperatura interna de uma casa é \SI{18}{\celsius}. Em um dado instante, a temperatura ambiente externa era de \SI{12}{\celsius} negativos. Ao sair do interior para o exterior dessa casa, qual é a variação de temperatura que uma pessoa experimenta?

\question Quantos anos foram transcorridos entre 520 a.C. e 450 a.C.?

\question O planeta Vênus é um dos mais quentes do nosso sistema solar, tendo uma temperatura média de \SI{450}{\celsius}. Enquanto isso, seu amigo Plutão é um dos mais frio, com temperaturas médias de \SI{250}{\celsius} negativos. Quantos graus Plutão é mais frio que Vênus?


\question No deserto de Gobi, localizado na Ásia, podem ser verificadas diferenças de temperatura de até \SI{60}{\celsius} entre o dia e a noite. Durante o dia, a temperatura chega a \SI{50}{\celsius}. A quanto chega a temperatura à noite?


\question Um avião levantou voo de uma cidade A que está a 50 metros acima do nível do mar. Subiu 300 metros, depois desceu 40 metros, subiu mais 80 metros, desceu até a metade da altura que estava, em relação ao nível do mar, então subiu mais 100 metros. Quanto precisará agora descer para chegar ao chão da cidade B, localizada a 30 metros acima do nível do mar?

\question O isolamento térmico de um avião permite suportar diferenças de temperatura de até \SI{60}{\degree C} seu interior e seu exterior. Mantendo a temperatura interna do avião em \SI{18}{\degree C}, qual é a mínima temperatura externa suportada?


\end{questions}